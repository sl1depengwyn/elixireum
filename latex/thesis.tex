\documentclass[oneside,final,14pt,a4paper]{extreport}

% set the compiler XeLatex to use fontspec package
\usepackage{fontspec}
 
\setmainfont{Times New Roman}

\usepackage{vmargin}
\setpapersize{A4}
\setmarginsrb{2.5cm}{2.2cm}{2.2cm}{2.2cm}{0pt}{10mm}{0pt}{13mm}
\usepackage{setspace}
\sloppy
\setstretch{1.5}
\usepackage{indentfirst}
\parindent=1.25cm

%%%%% ADDED TO SUPPORT TT BOLD FACES %%%%
\DeclareFontShape{OT1}{cmtt}{bx}{n}{<5><6><7><8><9><10><10.95><12><14.4><17.28><20.74><24.88>cmttb10}{}
\renewcommand{\ttdefault}{pcr}
%%%%% END %%%%%%%%%%%%%%%%%%%%%%%%%%%%%%% 
\usepackage{atbegshi,picture}
\usepackage[T1,T2A]{fontenc}
\usepackage[utf8]{inputenc}
\AtBeginShipout{\AtBeginShipoutUpperLeft{%
  \put(\dimexpr\paperwidth-1cm\relax,-1.5cm){\makebox[0pt][r]{\includegraphics[width=3cm]{figs/inno}}}%
}}


\usepackage[english]{babel}
\usepackage[backend=biber,style=ieee,autocite=inline]{biblatex}
\bibliography{ref.bib}
\DefineBibliographyStrings{english}{%
  bibliography = {References},}
\usepackage{blindtext}


\usepackage{pdfpages}
\newenvironment{bottompar}{\par\vspace*{\fill}}{\clearpage}
\usepackage{amsmath,amsfonts}
\usepackage{url}

\usepackage{amsthm}
\newtheorem{theorem}{Theorem}
\newtheorem{corollary}{Corollary}
\newtheorem{lemma}{Lemma}
\newtheorem{proposition}{Proposition}
\theoremstyle{definition}
\newtheorem{definition}{Definition}
\theoremstyle{remark}
\newtheorem*{remark}{Remark}
\theoremstyle{remark}
\newtheorem*{example}{Example}


\usepackage{titlesec}
\usepackage{float}
\usepackage{graphicx}
\graphicspath{{figs/}} %path to images
\usepackage{array}
\usepackage{multirow,array}
\usepackage{caption}
\usepackage{subcaption}
\usepackage{hyperref}
\usepackage{paralist}
\usepackage{listings}
\usepackage{zed-csp}
\usepackage{fancyhdr}
\usepackage{csquotes}
\usepackage{color}
% \usepackage{anyfontsize}
% \usepackage{mathptmx}
% \usepackage{t1enc}

\usepackage{chngcntr}
\usepackage{upgreek} 
\usepackage{bm}
\usepackage{hyperref}
\usepackage{setspace}
\usepackage{booktabs}
\usepackage{multirow}
\usepackage{longtable}
\usepackage[font=singlespacing, labelfont=bf]{caption}
\counterwithout{table}{chapter}
\renewcommand{\thetable}{\Roman{table}}
%Hints
\newcommand\pic[1]{(Fig. \ref{#1})} %Ref on figure
\newcommand\tab[1]{(Tab. \ref{#1})} %Ref on table

\setlength{\headheight}{32.0976pt}
\usepackage{enumitem}
\newlist{inlinelist}{enumerate*}{1}
\setlist*[inlinelist,1]{%
  label=(\arabic*),
}

\setcounter{secnumdepth}{4}
\captionsetup[table]{labelfont={normalfont}, name={TABLE}, labelsep={newline}}
\counterwithout{table}{chapter}
\renewcommand{\thetable}{\Roman{table}}
\setlength{\parindent}{2em} 
\DeclareCaptionLabelSeparator{figSep}{.\quad}
\captionsetup[figure]{labelfont={normalfont}, name={Fig.}, labelsep=period}
\counterwithout{figure}{chapter}

\titleformat{\section}[hang]{\fontsize{20}{24}\selectfont\filcenter}{\arabic{chapter}.\arabic{section}}{1em}{}
\titleformat{\subsection}[hang]{\itshape}{\Alph{subsection}.}{1em}{}[]
\titleformat{\subsubsection}[runin]{\itshape}{\arabic{subsubsection}}{1em}{}[$:$]
\titlespacing{\subsubsection}{1em}{1em}{1em}
\titleformat{\paragraph}[runin]{\itshape}{\alph{paragraph})}{1em}{}[$:$\quad]
\titlespacing{\paragraph}{2em}{1em}{1em}

\pagestyle{fancyplain}

% remember section title
\renewcommand{\chaptermark}[1]%
	{\markboth{\chaptername~\thechapter~--~#1}{}}

% subsection number and title
\renewcommand{\sectionmark}[1]%
	{\markright{\thesection\ #1}}

\rhead[\fancyplain{}{\bf\leftmark}]%
      {\fancyplain{}{\bf\thepage}}
\lhead[\fancyplain{}{\bf\thepage}]%
      {\fancyplain{}{\bf\rightmark}}
\cfoot{} %bfseries


\newcommand{\dedication}[1]
   {\thispagestyle{empty}
     
   \begin{flushleft}\raggedleft #1\end{flushleft}
}

\definecolor{commentgreen}{RGB}{2,112,10}
\definecolor{eminence}{RGB}{108,48,130}
\definecolor{weborange}{RGB}{255,165,0}
\definecolor{frenchplum}{RGB}{129,20,83}

\lstdefinelanguage{elixir}{
    morekeywords={case,catch,def,do,else,false,%
        use,alias,receive,timeout,defmacro,defp,%
        for,if,import,defmodule,defprotocol,%
        nil,defmacrop,defoverridable,defimpl,%
        super,fn,raise,true,try,end,with,%
        unless},
    otherkeywords={<-,->, |>, \%\{, \}, \{, \, (, )},
    sensitive=true,
    morecomment=[l]{\#},
    morecomment=[n]{/*}{*/},
    morecomment=[s][\color{purple}]{:}{\ },
    morestring=[s][\color{orange}]"",
    commentstyle=\color{commentgreen},
    keywordstyle=\color{eminence},
    stringstyle=\color{red},
	showstringspaces=false,
  captionpos=b
}

\lstset{numbers=left,xleftmargin=2em,frame=single,framexleftmargin=0em,numberstyle=\footnotesize\ttfamily}

\definecolor{verylightgray}{rgb}{.97,.97,.97}

\lstdefinelanguage{solidity}{
	keywords=[1]{anonymous, assembly, assert, balance, break, call, callcode, case, catch, class, constant, continue, constructor, contract, debugger, default, delegatecall, delete, do, else, emit, event, experimental, export, external, false, finally, for, function, gas, if, implements, import, in, indexed, instanceof, interface, internal, is, length, library, log0, log1, log2, log3, log4, memory, modifier, new, payable, pragma, private, protected, public, pure, push, require, return, returns, revert, selfdestruct, send, solidity, storage, struct, suicide, super, switch, then, this, throw, transfer, true, try, typeof, using, value, view, while, with, addmod, ecrecover, keccak256, mulmod, ripemd160, sha256, sha3}, % generic keywords including crypto operations
	keywordstyle=[1]\color{blue}\bfseries,
	keywords=[2]{address, bool, byte, bytes, bytes1, bytes2, bytes3, bytes4, bytes5, bytes6, bytes7, bytes8, bytes9, bytes10, bytes11, bytes12, bytes13, bytes14, bytes15, bytes16, bytes17, bytes18, bytes19, bytes20, bytes21, bytes22, bytes23, bytes24, bytes25, bytes26, bytes27, bytes28, bytes29, bytes30, bytes31, bytes32, enum, int, int8, int16, int24, int32, int40, int48, int56, int64, int72, int80, int88, int96, int104, int112, int120, int128, int136, int144, int152, int160, int168, int176, int184, int192, int200, int208, int216, int224, int232, int240, int248, int256, mapping, string, uint, uint8, uint16, uint24, uint32, uint40, uint48, uint56, uint64, uint72, uint80, uint88, uint96, uint104, uint112, uint120, uint128, uint136, uint144, uint152, uint160, uint168, uint176, uint184, uint192, uint200, uint208, uint216, uint224, uint232, uint240, uint248, uint256, var, void, ether, finney, szabo, wei, days, hours, minutes, seconds, weeks, years},	% types; money and time units
	keywordstyle=[2]\color{teal}\bfseries,
	keywords=[3]{block, blockhash, coinbase, difficulty, gaslimit, number, timestamp, msg, data, gas, sender, sig, value, now, tx, gasprice, origin},	% environment variables
	keywordstyle=[3]\color{violet}\bfseries,
	identifierstyle=\color{black},
	sensitive=true,
	comment=[l]{//},
	morecomment=[s]{/*}{*/},
	commentstyle=\color{gray}\ttfamily,
	stringstyle=\color{red}\ttfamily,
	morestring=[b]',
  captionpos=b,
	morestring=[b]"
}

% \lstset{
% 	language=Solidity,
% 	backgroundcolor=\color{verylightgray},
% 	extendedchars=true,
% 	basicstyle=\footnotesize\ttfamily,
% 	showstringspaces=false,
% 	showspaces=false,
% 	numbers=left,
% 	numberstyle=\footnotesize,
% 	numbersep=9pt,
% 	tabsize=2,
% 	breaklines=true,
% 	showtabs=false,
% 	captionpos=b
% }

\lstdefinelanguage{yul}{
  keywords={let, function, if, else, switch, case, default, for, break, continue, return},
  keywordstyle=\color{blue}\bfseries,
  ndkeywords={add, sub, mul, div, sdiv, mod, smod, addmod, mulmod, exp, signextend, lt, mstore8, mstore},
  ndkeywordstyle=\color{teal}\bfseries,
  comment=[l]{//},
  morecomment=[s]{/*}{*/},
  commentstyle=\color{gray}\ttfamily,
  stringstyle=\color{red}\ttfamily,
  morestring=[b]",
  morestring=[b]',
  sensitive=true,
  basicstyle=\ttfamily,
  numbers=left,
  % numberstyle=\tiny\color{gray},
  stepnumber=1,
  numbersep=10pt,
  tabsize=4,
  showspaces=false,
  showstringspaces=false,
  showtabs=false,
  % frame=single,
  captionpos=b,
  breaklines=true,
  breakatwhitespace=false,
  escapeinside={\%*}{*)}
}


\begin{document}

% \includepdf[pages=-]{title.pdf}
\tableofcontents
\listoftables
\listoffigures



% \newpage
% \begin{abstract}
% skip one line to make the abstract start with indent

В данной работе рассматривается необходимость создания нового языка программирования смарт-контрактов в экосистеме Ethereum, в которой в настоящее время доминируют контракты, написанные на Solidity. В нашей работе представлен Elixireum - функциональный язык с синтаксисом Elixir и динамической типизацией, предназначенный для повышения гибкости и креативности разработчиков децентрализованных приложений (dApp). Компилятор Elixireum был разработан и протестирован на совместимость с виртуальной машиной Ethereum (EVM) и способность поддерживать стандарты ERC-20 и ERC-721. Для сравнения потребления газа Elixireum и Solidity был использован комплексный набор тестов. Результаты показали, что, несмотря на функциональность и совместимость Elixireum с Ethereum, он потребляет больше газа, чем Solidity. Ключевые функции, такие как события и взаимодействие с хранилищем, были реализованы. Однако среди ограничений - высокий расход газа на развертывание, отсутствие внешних библиотек и взаимодействия с внешними смарт-контрактами. Будущая работа должна быть направлена на снижение затрат газа при развертывании и реализацию словарей, которые бы работали во время выполнения. Несмотря на высокое потребление газа, простота разработки и потенциал для более раннего развертывания Elixireum могут перевесить эти затраты, позиционируя его как жизнеспособную альтернативу в пространстве Web3.
\end{abstract}
% \setcounter{page}{8} % set manually an actual number from which introduction starts
% \setcounter{chapter}{5} % set manually an actual number from which introduction starts
\chapter{Introduction}
\label{chap:intro}

% 1. ethereum overview
%   - a popular blockhain platform created by Buterin in +-2016 with a big community of developers and crypto enthusiasts (link to some researches to show it's actuality and popularity), mb show actuality via daily market cap and other finance indicators.
%   - blockchain ecosystem, the core feature is dApps and smart contracts
%   - currenlty there are only 2 popular languages: Solidity and Vyper. And more than 90\% contracts written in solidity
%   - why it is important to write effective smart contracts (users pay for each EVM opcode)
%   - raise of rollups (about L2 rollups, almost zero costs and possibility to don't mind about gas consumption)
% 2. goal of the thesis (RQs hypoyhesis and blabla)
% 3. why elixireum is needed and what problems it solves
% 4. results 


This chapter serves as an introduction to the Ethereum ecosystem, highlighting the popularity of the platform and its role in the crypto community. 
Section \ref{sec:ethoverview} provides a brief overview of Ethereum history and evolution. Section \ref{sec:langdemand} underlines the demand for a new smart contract programming language and Section \ref{sec:goal} states the goal of the thesis. Section \ref{sec:thesoverview} describes other thesis chapters.


\section{Ethereum Overview}
\label{sec:ethoverview}

\subsection{Ethereum: An Innovative Blockchain Platform}
Ethereum, launched by Vitalik Buterin in July of 2015, is a major blockchain platform. It is supported by a large community of developers and cryptocurrency enthusiasts. To evaluate the importance and current popularity we can refer to CoinMarketCap\footnote{\href{https://coinmarketcap.com/}{CoinMarketCap - resources which provides analytics and prices for cryptocurrencies.}}. Ethereum ranks 2nd after Bitcoin, having the market cap more than \$380B and more than \$10B daily trading value, indicating the significant role of ETH in the digital economy.

\subsection{Blockchain Ecosystem and Core Features}
Ethereum supports decentralized applications (dApps) and smart contracts. These contracts automatically manage transactions and agreements without needing a middleman. This system finds an application in various types of applications in areas like finance and supply chain management.

\subsection{Programming Languages on Ethereum}

Solidity is the most used programming language on Ethereum \cite{SolidityWidelyUsed}. Vyper is the most popular alternative, however it shares a tiny part of the market. So, the scope of programming languages that can be used to develop decentralized applications is limited.

\subsection{Importance of Writing Effective Smart Contracts}
Using smart contracts on Ethereum costs money, known as "gas fees", which user i.e. transaction sender pays for each action a contract performs. These costs make it important to write efficient smart contracts to reduce expenses and improve performance.

\subsection{The Rise of Layer 2 Rollups}
With the increasing popularity and widespread adoption of Ethereum, the price of its native coin, ETH, has surged significantly. As the cost of ETH rises, even simple transactions like transferring coins can become too expensive, sometimes costing up to \$100 per transaction. Such high fees restrict the practical usability of the blockchain. Consequently, rollup blockchains have emerged to address these challenges. The concept behind rollup solutions can be described as a cheap blockchain (L2) on top of an expensive blockchain (L1). An L2 chain performs the execution of hundreds of transactions and then stores the resulting state changes within a single transaction on an L1. The fact that all the changes eventually are stored in an L1 blockchain enables an L2 chain to retain all L1 blockchain characteristics such as immutability and decentralization, while significantly reducing gas costs and increasing throughput. It allows developers to focus more on features and security, changing how dApps are made and used.


\section{Demand for a New Language}
\label{sec:langdemand}

DApp developers mostly use Solidity programming language. This language is imperative in nature. Monolingualism and monoparadigmism limit developers' flexibility and creativity. Additionally, Solidity has no support for certain features such as floating-point numbers, which are crucial for handling decimal values in calculations, thereby restricting its utility for a range of applications. Moreover, the high entry threshold associated with this language can deter newcomers from entering the field. It often requires a substantial investment of time and effort to become proficient, which can slow down the development process.

Given these constraints, there is a significant opportunity for a new language. A functional alternative, particularly one that supports dynamic typing and builds upon the foundations of a widely recognized programming language, could dramatically enhance the accessibility and efficiency of dApp development. Such a language would not only make it easier for new developers to adopt but also open up new possibilities for innovation in the Web3 space. This potential makes the development of a new language a promising avenue for expanding the capabilities and reach of the programming environment of Ethereum.

\section{Goal of the Thesis}
\label{sec:goal}

\subsection{Research Aims and Hypotheses}
Our goal is to create a smart contract language with Elixir syntax and dynamic types -- Elixireum. As a result, language should be expressive at least to be able to implement ERC-20\footnote{https://eips.ethereum.org/EIPS/eip-20} and ERC-721\footnote{https://eips.ethereum.org/EIPS/eip-721} tokens. We plan to add such features as immutability, macros support and pattern matching to make this language functional. Also, we expect that Elixireum will have the same or less gas consumption in comparison with Solidity.

\subsection{Expected Results}
\begin{itemize}
  \item \textbf{Working Compiler for Elixireum:} Development of a compiler capable of translating Elixireum-specific language into EVM bytecode.
  
  \item \textbf{Deployed ERC-20 Token:} Deployed ERC-20 token written on Elixireum as a showcase that Elixireum has fundamental functionality needed for smart contract development.
  
  \item \textbf{Compared Gas Consumption:} Analysis of gas usage for Elixireum contracts and comparative evaluation with Solidity.
\end{itemize}

\section{Thesis overview}
\label{sec:thesoverview}

This section overviews the thesis. Chapter 2 Literature Review contains an introduction to Ethereum and a review of the existing related literature. Chapter 3 Methodology includes a theoretical description of approaches which was used during the implementation phase. Chapter 4 Implementation provides a detailed description of the Elixireum implementation. Chapter 5 Results and Discussion shows the result we achieved and compares the Elixireum with other languages. Chapter 6 Conclusion concludes the research, and states the limitations and future work scope.


\chapter{Literature Review}
\label{chap:lr}
\chaptermark{Second Chapter Heading}


\section{Ethereum Virtual Machine}
Ethereum is a decentralized system, and at its core, the Ethereum Virtual Machine (EVM) is an essential component of the Ethereum blockchain. The EVM serves as the computational heart of the Ethereum network, enabling the execution of smart contracts and providing the foundation for a wide array of decentralized applications.

In the context of smart contract execution, it's important to mention the concept of gas limit. Gas limit is a critical aspect of Ethereum's execution model. Each operation and computation on the Ethereum Virtual Machine consumes a certain amount of gas, which is a measure of computational work. The gas limit is a cap set by the user or the entity initiating a contract execution, and it represents the maximum amount of gas they are willing to spend for that operation.

Gas limit, the EVM's stack-based architecture, and the wide set of available operations allow it to efficiently and securely for miners execute these smart contracts, granting it the status of a Turing-complete machine \cite{EthereumWhitepaper}. This level of computational flexibility empowers developers to create sophisticated programs.

\section{Existing EVM languages}
Official Ethereum documentation \cite{OfficialEthereumLanguages} enumerates only four languages that compiles to EVM: Solidity, Vyper, Yul and Fe. However, other attempts to create language for EVM are known \cite{CommunityEthereumLanguages}.

Among the available languages for EVM, Solidity stands out as the most widely adopted and utilized one \cite{SolidityWidelyUsed, SolidityDevsChallenges}. This dominance is further emphasized by data presented in \cite{SolidityVyperGithubUsage}, which reveals that there are approximately 800 times more Solidity files on Github compared to Vyper files. Consequently, it is evident that the majority of existing smart contracts on the Ethereum blockchain are written in Solidity. On one hand, the concentration of development and resources around a single language like Solidity can be viewed positively. It fosters a strong sense of community cohesion and ensures that the majority of developers are focused on refining and improving a single language, potentially enhancing its robustness and feature set. However, this dominance also brings significant challenges. The high prevalence of Solidity in the EVM ecosystem means that if a vulnerability or security flaw arises in the Solidity language, it affects a substantial portion of the smart contracts on the network. This centralized risk can be a double-edged sword, as a single point of failure in Solidity could have far-reaching consequences for the entire Ethereum blockchain. Furthermore, the Ethereum community is not without its criticisms of Solidity. Community members have voiced their concerns and challenges related to the language \cite{SolidityDevsChallenges}. These concerns range from the language's complexity to its lack of certain features, which may hinder the development of robust and secure smart contracts. So, decentralization and availability of choice is important, especially for such decentralized ecosystem as blockchain.

\subsection{Solidity}

\subsection{Vyper}

\section{Market Research}

\section{Compiler construction}

The process of compiler construction, is a crucial bridge between high-level programming languages and the machine code that EVM understands. This process can be divided into three or four main stages, each playing a pivotal role in transforming human-readable code into executable instructions that can be processed by the EVM. Here is a general overview of these stages:

\subsection{Tokenization}
  Tokenization is the initial step in the compilation process. During this stage, the source code is broken down into smaller units called tokens. These tokens are fundamental building blocks, including keywords, identifiers, operators, and constants. Tokenization simplifies the process of analyzing and parsing the code by providing a structured representation of the code's elements.

  Tokenization facilitates syntactical and lexical analysis, ensuring that the code adheres to the language's grammar and can be properly understood.

\subsection{Parsing}

  Parsing follows tokenization and focuses on the syntax of the source code. During parsing, the compiler checks the arrangement and structure of tokens to create a structured representation, often in the form of a syntax tree or abstract syntax tree (AST). This representation helps ensure that the code conforms to the language's grammar rules and can be further processed.

\subsection{Intermediate Representation (Optional)}

  While not always a mandatory stage, the use of an intermediate representation can greatly enhance the efficiency and capabilities of a compiler. In the context of EVM, various intermediate representations like Yul and Yul+ are used for optimizing code. These representations help in enhancing code quality, optimizing performance, and preparing the code for final code generation.

  Additionally, a language like Simplicity is employed for formal proofs. This stage is particularly important in scenarios where verification, security, and performance optimizations are key concerns.

\subsection{Code Generation}

  The final stage of the compilation process involves code generation. During this stage, the compiler generates the actual machine code that the EVM can execute. The generated code is specific to the EVM's hardware architecture and represents a translation of the original high-level code into instructions that the EVM can understand and execute.

  The code generation stage is crucial in ensuring that the compiled code is efficient and correctly reflects the intended behavior of the high-level source code.


In conclusion, compiler construction is fundamental to the Ethereum ecosystem, enabling the translation of high-level languages into machine code that can be executed on the EVM. The stages of tokenization, parsing, optional intermediate representation, and code generation are essential components of this process. The flexibility to compile existing languages like Vyper further enriches the Ethereum development landscape, making it accessible to a wider range of developers and promoting the platform's growth and adoption.

\section{The Elixir language}

Elixir is a modern programming language, which is working on the top of Erlang VM. 

it's not popular as python or Javascript but it has own community, and it is in top 100 languages by TIOBE \cite{TIOBE}

To proceed with design of new language for smart contracts, we decided to choose elixir as a base. Because choice of existing language
1. will lower entry threshold into new language.
2. will attract Elixir developers to the Ethereum
3. allows to utilize existing compiler components such as lexer and parser

Such approach we can met in Vyper language which is built on the top of Python languag
\section{Conclusion}

% \newpage


% \begin{longtable}{c|c}
% \caption[This is the title I want to appear in the List of Tables]{Simulation Parameters} \label{table:secsimulation_params} \\
% \hline
% A & B  \\
% \hline
% \endfirsthead
% \multicolumn{2}{@{}l}{} \\
% \hline
% A & B \\
% \hline
% \endhead
% \hline
%  \textbf{Parameter} & \textbf{Value}\\
%  \hline
%  Number of vehicles & $|\mathcal{V}|$\\
%  \hline
%  Number of RSUs & $|\mathcal{U}|$\\
%  \hline
%  RSU coverage radius & 150 m\\
%  \hline
%  V2V communication radius & 30 m\\
%  \hline
%  Smart vehicle antenna height & 1.5 m\\
%  \hline
%  RSU antenna height & 25 m\\
%  \hline
%  Smart vehicle maximum speed & $v_{max}$ m/s\\
%  \hline
%  Smart vehicle minimum speed & $v_{min}$ m/s\\
%  \hline
%  Common smart vehicle cache capacities & $[50, 100, 150, 200, 250]$ mb\\
%  \hline
%  Common RSU cache capacities & $[5000,1000,1500,2000,2500]$ mb\\
%  \hline
%  Common backhaul rates & $[75, 100, 150]$ mb/s\\
%  \hline
% \end{longtable}

% \begin{figure}[hbt]
% \centering
% \includegraphics[]{figs/inno.png}
% \caption{One kernel at $x_s$ (\emph{dotted kernel}) or two kernels at
% $x_i$ and $x_j$ (\textit{left and right}) lead to the same summed estimate
% at $x_s$. This shows a figure consisting of different types of
% lines. Elements of the figure described in the caption should be set in
% italics, in parentheses, as shown in this sample caption.}
% \label{fig:secex}
% \end{figure}

% This description implies several essential properties of the task at hand:
% \begin{enumerate}
%     \item Watermark must contain all necessary information, but still, be placeable and recognizable even on smaller images. The produced watermark must be compact but have the possibility to store enough information. 
%     \item To prevent easy tampering, the watermark must be invisible to the naked eye (and, preferably, to basic image parsing tools). If malefactor does not know about the existence of watermark, they might not even try to remove it and disable it. 
% \end{enumerate}

\chapter{Methodology}
\label{chap:met}
\chaptermark{Third Chapter Heading}

Chapter overview

\section{Compilers theory}
\label{}

\section{Problems (no researches, more hacking way to research)}
\label{}

\section{Research design (methods: techonology stack, compiler stages, decisions like why we compile and not interpret)}
\label{}

\section{Limitations (elixirish distribution is not applicable to evm, so it is cannot be used in smart contracts, libraries are out of the scope, external smart contract interactions, need to spec types for public functions, need of solidity compiler )}
\label{}

\section{architecture and modules overview}
\label{}

\section{other languages comparison}
\label{}

\section{Conclusion}



\ldots

Referencing other chapters \ref{chap:lr}, \ref{chap:met}, \ref{chap:impl}, \ref{chap:eval} and \ref{chap:conclusion}
\begin{longtable}{c|c}
\caption[This is the title I want to appear in the List of Tables]{Simulation Parameters} \label{table:thisimulation_params} \\
\hline
A & B  \\
\hline
\endfirsthead
\multicolumn{2}{@{}l}{} \\
\hline
A & B \\
\hline
\endhead
\hline
 \textbf{Parameter} & \textbf{Value}\\
 \hline
 Number of vehicles & $|\mathcal{V}|$\\
 \hline
 Number of RSUs & $|\mathcal{U}|$\\
 \hline
 RSU coverage radius & 150 m\\
 \hline
 V2V communication radius & 30 m\\
 \hline
 Smart vehicle antenna height & 1.5 m\\
 \hline
 RSU antenna height & 25 m\\
 \hline
 Smart vehicle maximum speed & $v_{max}$ m/s\\
 \hline
 Smart vehicle minimum speed & $v_{min}$ m/s\\
 \hline
 Common smart vehicle cache capacities & $[50, 100, 150, 200, 250]$ mb\\
 \hline
 Common RSU cache capacities & $[5000,1000,1500,2000,2500]$ mb\\
 \hline
 Common backhaul rates & $[75, 100, 150]$ mb/s\\
 \hline
\end{longtable}

\begin{figure}[hbt]
\centering
\includegraphics[]{figs/inno.png}
\caption{One kernel at $x_s$ (\emph{dotted kernel}) or two kernels at
$x_i$ and $x_j$ (\textit{left and right}) lead to the same summed estimate
at $x_s$. This shows a figure consisting of different types of
lines. Elements of the figure described in the caption should be set in
italics, in parentheses, as shown in this sample caption.}
\label{fig:thiex}
\end{figure}


\ldots
\chapter{Design and Implementation}
\label{chap:impl}

This chapter outlines the principal elements of the design of Elixireum compiler. It overviews the technology stack employed in the development process and underscores the pivotal decisions that shaped the architecture of the compiler.

% preliminary structure:

% 1. architecture
% 2. technical details (memory included)
% 3. modules details


% overview of architecture (.exm -> lexer -> parser -> compiler (code gen) -> .yul code -> solidity compiler -> EVM bytecode)
%                                                            |
%                                                            -> ABI generator

% detailed overview of modules (for example: calldata parsing, compiler (two passes), return data transformer, storage)


% technical details:
%   - memory organization
%     - how variables are stored, dynamic typing
%     - immutability of variables
%     - offset
%   - show events/module args syntax
%   - key features
%     - events
%     - constructor
%     - storage
%     - ABI for public functions
%     - reverts with utf-8 error string
%   - standard functions declaration on demand (transitivity included) and mechanism itself (module responsible for )

\section{Stack overview}
\label{sec:architecture}
This section rationalize the technology stack used in the project: Elixir, Yul, Solidity compiler.
\subsection{Choice of Elixir}
Here is the rationalization behind the choice of Elixir:
\begin{itemize}
    \item One of the most popular open source blockchain explorers, Blockscout\footnote{\href{https://www.blockscout.com/}{https://www.blockscout.com} -- Official Blockscout website}, is written in Elixir. From this we can say that Elixir is well heard of in Web3 community.
    \item Our team possesses substantial expertise in this programming language.
    \item Elixir is open source so we can easily reuse its machinery for tokenization and parsing Elixireum.
\end{itemize}

\subsection{Choice of Yul}
\begin{itemize}
    \item Initially, our strategy was to compile Elixireum directly into Ethereum Virtual Machine (EVM) bytecode. Subsequently, we pivoted towards an intermediate representation (IR), selecting Yul as IR aligned with our requirements. Advantages from this decision are as follows:
        \begin{itemize}
            \item Compatibility with all EVM versions, eliminating the need for version-specific concerns. 
            \item Avoidance of reimplementing primitives such as function calls, variable and stack management, thereby freeing up resources to enhance the feature set of the language.
            \item Reduced gas consumption facilitated by the optimizer of Solidity compiler.
        \end{itemize}
      Despite these considerations, the choice comes with limitations, notably:
        \begin{itemize}
            \item Dependency on the Solidity compiler for Elixireum.
            \item The unresolved bug\footnote{\href{https://github.com/ethereum/solidity/issues?q=is:issue+is:open+StackTooDeepError}{https://github.com/ethereum/solidity/issues?q=is:issue+is:open+StackTooDeepError}} in the Solidity compiler that appears for large smart contracts written in Elixireum when compiled with the optimizer disabled. The bug results in an inability to operate with a large number of variables or deep stack depths.
        \end{itemize}
      No significant downsides compared to direct compilation to EVM bytecode, since bytecode can be used in Yul directly.

    \item We also considered mapping BEAM (Erlang VM) bytecode to EVM bytecode. However, this option presented significant challenges due to the fundamental architectural and philosophical differences between the two virtual machines. BEAM is specifically tailored for environments that demand high concurrency, distribution, and fault tolerance, and it includes a set of opcodes optimized for these conditions. In contrast, the EVM is designed to run smart contracts within blockchain ecosystems, prioritizing deterministic execution and gas metering to protect against spam and enhance network security. BEAM incorporates specialized instructions for concurrency and distributed processing that do not directly translate to the EVM, because of these distinct focuses.
\end{itemize}

\section{Technical details}

\subsection{Memory organization}
\label{sec:memory_organization}

In our language implementation, we decided to store all runtime variables in the volatile memory of the Ethereum Virtual Machine. This decision is justified by the low gas cost associated with memory operations and the simplicity of implementation. We considered two alternatives: transient storage, which is more expensive than memory, and the stack, which involves more complex management.

In the generated Yul code, a variable is defined as a pointer to the memory location where its value is stored. At this pointer, the first byte is reserved for the type number, and the subsequent bytes contain the value itself for simple cases. For more complex cases, such as lists and tuples, the first byte still indicates the type, followed by a word\footnote{in this context word is 32 bytes} that specifies the count of elements, and then the elements themselves, each stored as an inline variable (type + value). For strings or byte arrays, the first byte indicates the type (1 for strings, 102 for byte arrays), the next word specifies the byte count, followed by the actual bytes.

Since we developed a functional language, one of its fundamental principles is immutability. This means that each assignment operation creates a new variable. In practice, we "forget" the previous pointer, assign a new pointer, and place the variable in a new location. Each new variable should be stored sequentially after the previous one. To achieve this, we define the offset\$ variable in the generated Yul code. This variable holds the memory address where the next variable will be stored. As a workaround for corner cases, we update this variable using the msize() call, assigning the current memory size rounded to one word to the offset\$ variable.

In Elixireum syntax, there are no type specifiers for variables. To handle dynamic types, we established several principles upon which our type system is built:
\begin{itemize}
  \item Return values of public functions must strictly match the specified types; otherwise, a revert\footnote{opcode which stops the current execution and throw an error. All the state changes are reverted} operation will be triggered.
  \item For literals, the type is defined at compile time.
  \item The type is stored in the first byte of the variable.
  \item We defined a standard cast function that developers can use in Elixireum code.

  The following code snippet demonstrates the usage of the cast function:
  \begin{lstlisting}[caption={Example of cast function usage}, language=elixir, label={lst:contract_structure}]
    @spec decimals() :: Types.UInt8.t()
    def decimals() do
      Utils.cast(18, Types.UInt8)
    end  
  \end{lstlisting}
\end{itemize}

The following Elixireum code illustrates a simple case:

\begin{lstlisting}[caption={Elixireum code for simple case}, language=elixir]
  a = 1
  b = 2
\end{lstlisting}

This Elixireum code is translated into Yul code as follows:

\begin{lstlisting}[caption={Generated yul code for simple case}, language=yul]
  let offset$ := msize()      // define offset$, set it to current memory size 
  let a$ := offset$           // define variable as the memory pointer
  mstore8(offset$, 67)        // 67 is int256 type, store it as one byte
  offset$ := add(offset$, 1)  // increase current offset by 1 byte, which is used by type on the line above
  mstore(offset$, 1)          // store the value itself
  offset$ := add(offset$, 32) // int256 value takes 32 bytes

  let b$ := offset$           // define variable as the memory pointer
  mstore8(offset$, 67)        // 67 is uin256 type
  offset$ := add(offset$, 1)  // increase current offset by 1 byte, which is used by type on the line above
  mstore(offset$, 2)          // store the value itself
  offset$ := add(offset$, 32) // int256 value takes 32 bytes
\end{lstlisting}

In this Yul code, offset\$ is initialized to the current memory size. Each variable (a\$ and b\$) is defined as a memory pointer. The type of each variable is stored using mstore8, and the actual value is stored using mstore. The offset is incremented appropriately to account for the type byte and the 32-byte value for each variable.

The following table shows a snippet of the EVM memory for this simple case:

\begin{table}[h!]
  \centering
  \renewcommand{\arraystretch}{1.2} % Reduces the height of the table
  \begin{tabular}{|>{\centering\arraybackslash}m{2cm}|>{\centering\arraybackslash}m{1cm}|>{\centering\arraybackslash}m{1cm}|>{\centering\arraybackslash}m{1cm}|>{\centering\arraybackslash}m{0.75cm}|>{\centering\arraybackslash}m{1cm}|>{\centering\arraybackslash}m{1cm}|>{\centering\arraybackslash}m{1cm}|>{\centering\arraybackslash}m{1cm}|>{\centering\arraybackslash}m{0.75cm}|>{\centering\arraybackslash}m{1cm}|}
  \hline
  \textbf{Addresses} & 0x00 & 0x01 & 0x02 & ... & 0x21 & 0x22 & 0x23 & 0x24 & ... & 0x42 \\ \hline
  \textbf{Values}    & 0x43 & 0x00 & 0x00 & ... & 0x01 & 0x43 & 0x00 & 0x00 & ... & 0x01 \\ \hline
  \end{tabular}
  \caption{EVM Memory snippet for simple case}
  \label{tab:evm_memory}
  \end{table}

An example of handling strings in Elixireum is as follows:

\begin{lstlisting}[caption={Elixireum code for string case}, language=elixir]
  a = "abc" # hex representation of this utf-8 string is 0x616263
\end{lstlisting}
  
The corresponding Yul code generated for this string case is:

\begin{lstlisting}[caption={Generated yul code for string case}, language=yul]
  let offset$ := msize()      // define offset$, set it to current memory size 
  let a$ := offset$           // define variable as the memory pointer
  mstore8(offset$, 1)         // store the type, 1 is the string type
  offset$ := add(offset$, 1)  // increase current offset by 1 byte, which is used by type on the line above
  mstore(offset$, 3)          // store the byte size of the string
  offset$ := add(offset$, 32) // increase current memory pointer by 32 bytes, since the size of the string is 256 bit lenght
  mstore(offset$, 0x6162630000000000000000000000000000000000000000000000000000000000) \\ store the string itself
  offset$ := add(offset$, 3)  // string has length only 3 bytes
\end{lstlisting}

In this Yul code, the variable a\$ is defined as a memory pointer. The type of the variable is stored using mstore8, followed by storing the byte size of the string using mstore. The offset is incremented by the size of the string (3 bytes) after storing the string itself.

The following table illustrates a snippet of the EVM memory for this string case:

\begin{table}[h!]
  \centering
  \renewcommand{\arraystretch}{1.2} % Reduces the height of the table
  \begin{tabular}{|>{\centering\arraybackslash}m{2cm}|>{\centering\arraybackslash}m{1cm}|>{\centering\arraybackslash}m{1cm}|>{\centering\arraybackslash}m{1cm}|>{\centering\arraybackslash}m{0.75cm}|>{\centering\arraybackslash}m{1cm}|>{\centering\arraybackslash}m{1cm}|>{\centering\arraybackslash}m{1cm}|>{\centering\arraybackslash}m{1cm}|}
  \hline
  \textbf{Addresses} & 0x00 & 0x01 & 0x02 & ... & 0x20 & 0x21 & 0x22 & 0x23 \\ \hline
  \textbf{Values}    & 0x01 & 0x00 & 0x00 & ... & 0x03 & 0x61 & 0x62 & 0x63 \\ \hline
  \end{tabular}
  \caption{EVM Memory snippet for string case}
  \label{tab:evm_memory}
  \end{table}

An example of handling lists in Elixireum is as follows:

\begin{lstlisting}[caption={Elixireum code for list case}, language=elixir]
  a = [1, true, "abc"]
\end{lstlisting}
  
\begin{lstlisting}[caption={Generated yul code for string case}, language=yul]
  let offset$ := msize()
  
  // define int256 literal
  let var0$ := offset$
  ...  
  // define bool literal
  let bool_var1$ := offset$
  ...
  // define string literal
  let str2$ := offset$
  ...
  
  // define the list itself
  let a$ := offset$ // define variable as the memory pointer
  mstore8(offset$, 103) // store the list type
  offset$ := add(offset$, 1)
  mstore(offset$, 3) // store the length of the list 
  offset$ := add(offset$, 32) // increase current memory pointer by 32 bytes, since the size of the list is 256 bit lenght
  let ignored_
  ignored_, offset$ := copy_from_pointer_to$(var0$, offset$) // copy int256 variable to the start of the list & update offset$
  ignored_, offset$ := copy_from_pointer_to$(bool_var1$, offset$) // copy bool variable to the list & update offset$
  ignored_, offset$ := copy_from_pointer_to$(str2$, offset$) // copy string variable to the list & update offset$
\end{lstlisting}

In this Yul code the variables var0\$, bool\_var1\$, and str2\$ are defined for the integer, boolean, and string literals, respectively. Then list a\$ is defined. The length of the list is stored using mstore, and the actual values of the list elements are copied into memory using the copy\_from\_pointer\_to\$ function, which updates the offset accordingly.

The following table illustrates a snippet of the EVM memory for this list case:
\begin{table}[h!]
  \centering
  \renewcommand{\arraystretch}{1.2} % Reduces the height of the table
  \begin{tabular}{|>{\centering\arraybackslash}m{3cm}|>{\centering\arraybackslash}m{1cm}|>{\centering\arraybackslash}m{1cm}|>{\centering\arraybackslash}m{1cm}|>{\centering\arraybackslash}m{1cm}|>{\centering\arraybackslash}m{1cm}|>{\centering\arraybackslash}m{1cm}|>{\centering\arraybackslash}m{1cm}|>{\centering\arraybackslash}m{1cm}|>{\centering\arraybackslash}m{1cm}|>{\centering\arraybackslash}m{1cm}|}
  \hline
  \textbf{Addresses} & ... & 0x47 & 0x48 & ... & 0x67 & 0x68 & 0x69 & ... & 0x88 & 0x89 \\ \hline
  \textbf{Values}    & ... & 0x67 & 0x00 & ... & 0x03 & 0x43 & 0x00 & ... & 0x01 & 0x02 \\ \hline \hline
  \textbf{Addresses} & 0x8a & 0x8b & 0x8c & ... & 0xab & 0xac & 0xad & 0xae & ... & ... \\ \hline
  \textbf{Values}    & 0x01 & 0x01 & 0x00 & ... & 0x03 & 0x61 & 0x62 & 0x63 & ... & ... \\ \hline
  \end{tabular}
  \caption{EVM Memory snippet for list case}
  \label{tab:evm_memory_aux} 
  \end{table}

\subsection{Key features}
\label{sec:key_features}

\begin{enumerate}
  \item \textbf{Events\footnote{\url{https://docs.soliditylang.org/en/v0.8.25/abi-spec.html\#events}}}

  Events are essential for highlighting storage changes or other significant events in the lifecycle of a smart contract. They can be easily retrieved from the blockchain after being emitted. Events can be defined and emitted in Elixireum as follows:

  \begin{lstlisting}[caption={Events}, language=elixir, label={lst:calldata_decoding_recursive}]
  @test name: "Test", indexed_arguments: [addr: Types.Address.t()], data_arguments: [value: Types.UInt256.t()]
    ...
  addr = ~ADDRESS(0x0000000000000000000000000000000000000000)
  value = 100
  Event.emit(@test, [addr: addr value: value])
  \end{lstlisting}
  
  \item \textbf{Storage}
  
  Each address in Ethereum has its own associated persistent storage, which functions as a key-value store with 32-byte keys and values. This storage is crucial for maintaining a global state. In Elixireum, storage variables can be specified as a module argument using the @ symbol. The type of the storage variable can be any available type, a mapping, or an inner mapping of any depth.

  \begin{lstlisting}[caption={Storage}, language=elixir, label={lst:calldata_decoding_recursive}]
    @balances type: %{Types.Address.t() => Types.UInt256.t()}
    @totalSupply type: Types.UInt256.t()
    ...
    Storage.store(@totalSupply, 1_000_000) # store the value to the storage
    supple = Storage.get(@totalSupply)     # get the value from the storage
    
    addr_1 = ... # define some address
    Storage.store(@balances[addr_1], 1_000)  # update the @balances mapping for addr_1
    balance = Storage.get(@balances[addr_1]) # get value for addr_1 from @balances
  \end{lstlisting}
  
  \item \textbf{Constructor}

  The constructor is called once, during the deployment of the smart contract. It is used to initialize the contract, set the owner, or define any other storage variables. Essentially, it executes any necessary operations upon contract creation.
  
  \begin{lstlisting}[caption={Constructor}, language=elixir, label={lst:calldata_decoding_recursive}]
    @spec constructor(Types.String.t(), Types.String.t()) :: nil
    def constructor(name, symbol) do
      Storage.store(@name, name)
      Storage.store(@symbol, symbol)
      Storage.store(@owner, Blockchain.caller())
    end
  \end{lstlisting}
  
  \item \textbf{ABI for Public Functions}
  
  The compiler enforces the specification of all public functions through compile-time errors. After compilation, all defined specifications are converted to the ABI (Application Binary Interface) JSON, which is necessary for correctly calling the smart contract after deployment.

  \begin{lstlisting}[caption={@spec}, language=elixir, label={lst:calldata_decoding_recursive}]
  @spec a(Types.Int256.t()) :: Types.String.t()
  def a(_int) do
    "return string"
  end
  \end{lstlisting}
  
  Resulting JSON ABI:
  \begin{lstlisting}[caption={ABI }, language=json, label={lst:calldata_decoding_recursive}]
    [
      {
          "name": "a",
          "type": "function",
          "outputs": [
              {
                  "name": "output",
                  "type": "string",
                  "internal_type": "string"
              }
          ],
          "inputs": [
              {
                  "name": "_int",
                  "type": "int256",
                  "internal_type": "int256"
              }
          ],
          "stateMutability": "view"
      }
  ]
  \end{lstlisting}
  
  \item \textbf{Reverts with UTF-8 Error String}

  Developers can revert the execution of a smart contract, optionally specify an error string. The revert reason for the transaction will be a hex-encoded UTF-8 string.

  \begin{lstlisting}[caption={Reverts}, language=elixir, label={lst:calldata_decoding_recursive}]
    raise "ERC20InvalidSender(address(0))"
  \end{lstlisting}
\end{enumerate}

\section{Modules overview}
This section provides a detailed overview of the main modules of the compiler. The modules are listed in the order corresponding to the contract life cycle i.e., deployment and then calls.
\subsection{Yul compiler}
The Yul compiler calls the Solidity compiler to generate bytecode, given the path to the standard JSON input file\footnote{\url{https://docs.soliditylang.org/en/v0.8.25/using-the-compiler.html\#input-description}}. For this purpose, it looks for the Solidity compiler that is capable of compiling Yul to bytecode. If there is no Solidity compiler already downloaded, the Yul compiler module downloads platform specific Solidity compiler. Then, it returns standard JSON output\footnote{\url{https://docs.soliditylang.org/en/v0.8.25/using-the-compiler.html\#output-description}}.
\subsection{Compiler}
\label{ssec:compiler}


\begin{figure}[h!]
  \centering
  \includegraphics[width=0.8\textwidth]{figs/arch.png}
  \caption{Elixireum compiler architecture}
  \label{fig:arch}
\end{figure}

Here is the pipeline of actions performed by the main Compiler module, each pipeline step is performed only if all the previous steps are successful.

\begin{itemize}
    \item Reading the source code from the file system.
    \item Running the Elixir tokenizer and parser on the source code.
    \item Performing the first pass of the Elixir AST in pre-order to gather information about defined functions, storage variables, and events, and packing this data into the following structure:
    
    \begin{lstlisting}[caption={Contract structure}, language=elixir, label={lst:contract_structure}]
      @type t :: %Contract{
          functions: Functions.t(),
          name: String.t(),
          private_functions: Functions.t(),
          variables: %{atom() => Variable.t()},
          events: %{atom() => Event.t()},
          aliases: %{atom() => list()}
        }
    \end{lstlisting}
    
    At this step, the compiler checks that public defined functions have corresponding typespecs\footnote{Typespec is the type specification of the function used in Elixir}, storage variables have types and events have all required fields.

    \item Perform the second traversal of the Elixir AST in post-order to generate Yul code.
    
    Specifically, this step generates a special function called a constructor. This function is executed at the deployment stage, and it takes its arguments not from calldata, but from the code. The Yul code that decodes arguments is generated using the calldata decoding module.

    Then, the compiler generates a function selector. The function selector is used to identify which function is being called based on the first four bytes of the call data. The first four bytes of the call data should be equal to the first four bytes of the Keccak256 hash of the string in the following format:
    \lstinline|function_name(1st_arg_type_name,2nd_arg_type_name,...)|
    constructed for the calling function, this is called method ID. For example, if the function has the following signature 
    \lstinline|transfer(address to, uint256 value)|
    Keccak256 is performed on 
    \lstinline|transfer(address,uint256)| The result of Keccak 256 is equal to 0xa9059cbb2ab0\dots So, the method ID for the transfer function is 0xa9059cbb. The function selector is simply a switch case expression where each public defined function corresponds to a case with its method ID. For each case statement, the compiler generates Yul code, responsible for function arguments decoding from call data, using the call data decoding module. Then the function call is generated as if it was a private function. Then, the compiler generates the code that encodes the value returned by the function into the return format using the return data encoding module.

    Then, the compiler generates Yul code for all user-defined functions, both public and private, since the difference between public and private functions is factored out into the function selector.

    Then, the compiler generates Yul code for standard functions used in user-defined functions. Standard functions are represented by the following structure:

    \begin{lstlisting}[caption={Standard function structure}, language=elixir, label={lst:standard_function_structure}]
      @type t :: %__MODULE__{
          deps: %{atom() => t()},
          yul: String.t()
        }
    \end{lstlisting}

    The field deps represents other standard functions used by this standard function. This mechanism allows the definition of standard functions on demand, thus reducing the size of the Yul code and lowering the cost of deployment. Here is the mechanism that resolves dependencies:


    \begin{lstlisting}[caption={Standard function dependencies}, language=elixir, label={lst:standard_function_dependencies}]
    defp generate_std_functions(used_std_functions, definitions \\ %{}) do
      used_std_functions |>
      Enum.reduce(definitions, fn {function_name,
                                  %StdFunction{yul: yul, deps: deps}},
                                  definitions_acc ->
        {_, not_defined_deps} = Map.split(deps, Map.keys(definitions_acc))

        generate_std_functions(not_defined_deps, Map.put_new(definitions_acc, function_name, yul))
      end)
    end
    \end{lstlisting}
      
    
    \item Then, using the output of the first pass, the compiler generates ABI via ABI generator module.

\end{itemize}

\subsection{ABI generator}
\label{ssec:abi_generator}
The ABI generator takes the contract information collected by the compiler in the form of the contract structure \ref{lst:contract_structure} and generates a JSON ABI using the functions and events fields of the contract structure.


\subsection{Calldata decoding}
\label{ssec:calldata_decoding}
The call data decoding module recursively generates code for parsing arguments using information from the typespec of the function. The main function of this module is the decode function. It is generalized in terms of functions that perform loading and copying of data in order to reuse this function not only for argument decoding in calls, but also for argument decoding in the constructor where arguments are stored in memory, not in calldata. Here is an example of the base case of the decode function:

\begin{lstlisting}[caption={Calldata decoding base case}, language=elixir, label={lst:calldata_decoding_base}]
  def decode(
    _arg_name,
    %Type{encoded_type: encoded_type} = type,
    data_load_fn,
    _data_copy_fn,
    calldata_var,
    _init_calldata_var
  )
  when encoded_type > 69 and encoded_type < 102 do
    """
    mstore8(memory_offset$, #{type.encoded_type})
    memory_offset$ := add(memory_offset$, 1)
    mstore(memory_offset$, #{data_load_fn}(#{calldata_var}))
    memory_offset$ := add(memory_offset$, #{type.size})
    #{calldata_var} := add(#{calldata_var}, 32)
    """
  end
\end{lstlisting}

This function clause generates code that copies the value from calldata to memory and store the type for the Bytes<N> ABI type.

Here is an example of the recursive case of the decode function:

\begin{lstlisting}[caption={Calldata decoding recursive case}, language=elixir, label={lst:calldata_decoding_recursive}]
  def decode(
        arg_name,
        %Type{
          encoded_type: 3,
          components: components,
          size: size
        } = type,
        data_load_fn,
        data_copy_fn,
        calldata_var,
        init_calldata_var
      ) do
    tail_offset_var_name = "#{calldata_var}$#{arg_name}"
    init_tail_offset_var_name = "#{init_calldata_var}$#{arg_name}_init"

    """
    mstore8(memory_offset$, #{type.encoded_type})
    memory_offset$ := add(memory_offset$, 1)
    mstore(memory_offset$, #{Enum.count(components)})
    memory_offset$ := add(memory_offset$, 32)

    #{if size == :dynamic do
      """
      let #{tail_offset_var_name} := add(#{init_calldata_var}, #{data_load_fn}(#{calldata_var}))
      """
    else
      """
      let #{tail_offset_var_name} := #{calldata_var}
      """
    end}

    let #{init_tail_offset_var_name} := #{tail_offset_var_name}

    #{for {component, index} <- Enum.with_index(components) do
      decode(arg_name <> "#{index}", component, data_load_fn, data_copy_fn, tail_offset_var_name, init_tail_offset_var_name)
    end}
    """
  end
\end{lstlisting}

This function clause recursively generates code that copies the value from calldata to memory and stores type for the tuple ABI type. However, it only considers the tuple structure, i.e., the number of elements in it, and not the types of the elements, each element is decoded recursively.
  
\subsection{Emitting event}

The Event module is responsible for preparing and emitting events. Function emit, which is the main function of the module, takes the event as argument in the following form:

\begin{lstlisting}[language=elixir, caption={Event structure}, label={lst:event_structure}]
  @type t :: %__MODULE__{
          name: atom(),
          indexed_arguments: [keyword(Type.t())],
          data_arguments: [keyword(Type.t())],
          keccak256: String.t()
        }
\end{lstlisting}
  
An event has its signature, similar to a method ID for functions, but an event signature is the full Keccak256 hash of the following format:
$$event\_name(1st\_arg\_type\_name,2nd\_arg\_type\_name,...)$$

Events have indexed and data arguments. Indexed arguments are placed in the topics of the log. Topics are used to filter logs when fetching them from the blockchain. However, a log can have only up to four topics, with the first one being the event signature, so an event can have up to three indexed arguments. The size of the topics is limited to 32 bytes. For values larger than 32 bytes, its Keccak256 hash is used. The other data could be stored as data arguments that are placed in the data field of the log, with any value stored as is. The module generates code for encoding Elixireum values into indexed and data arguments. The function encode\_indexed\_argument is responsible for encoding indexed arguments. Here is the clause for types that are smaller than 32 bytes and are placed in topics as is:

\begin{lstlisting}[language=elixir, caption={Encode word-size indexed argument}, label={lst:encode_indexed_argument}]
  defp encode_indexed_argument(
    arg_name,
    %Type{encoded_type: encoded_type} = type,
    %YulNode{yul_snippet_usage: yul_snippet_usage},
    compiler_state
  )
  when encoded_type not in [1, 3, 102, 103] do
    var_name = "indexed_#{arg_name}_keccak_var$"
    arg_name_pointer = "indexed_#{arg_name}_pointer$"

    {"""
    let #{arg_name_pointer} := #{yul_snippet_usage}

    #{Utils.generate_type_check(arg_name_pointer, encoded_type, "Wrong type for indexed argument #{arg_name}", compiler_state.uniqueness_provider)}

    #{arg_name_pointer} := add(#{arg_name_pointer}, 1)
    let #{var_name} := shr(#{8 * (32 - type.size)}, mload(#{arg_name_pointer}))
    """, var_name}  
  end
\end{lstlisting}

Here is the clause for types that do not fit in one word:

\begin{lstlisting}[language=elixir, caption={Encode compex type indexed argument}, label={lst:encode_indexed_argument}]
  defp encode_indexed_argument(
         arg_name,
         %Type{} = type,
         %YulNode{yul_snippet_usage: yul_snippet_usage},
         _compiler_state
       ) do
    init_var_name = "indexed_#{arg_name}_keccak_init$"
    var_name = "indexed_#{arg_name}_keccak_var$"
    arg_name_pointer = "indexed_#{arg_name}_pointer$"

    {"""
     let #{init_var_name} := offset$
     let #{arg_name_pointer} := #{yul_snippet_usage}
     #{do_encode_indexed_argument(type, arg_name_pointer, init_var_name, 0)}
     let #{var_name} := keccak256(#{init_var_name}, sub(offset$, #{init_var_name}))
     """, var_name}
  end
\end{lstlisting}

The function do\_encode\_indexed\_argument is used to convert and copy Elixireum values to the format for Keccak256 and subsequent use in topics. It is defined recursively, similar to the calldata decode function.


The function encode\_data\_argument generates code that encodes data arguments, it uses the Return data encoding module \ref{ssec:return_data_encoding}. 

\begin{lstlisting}[language=elixir, caption={Encode data argument}, label={lst:encode_data_argument}]
  defp encode_data_argument(
         arg_name,
         %Type{} = type,
         %YulNode{yul_snippet_usage: yul_snippet_usage},
         index
       ) do
    """
    let #{arg_name}_$ := processed_return_value$
    let #{arg_name}_init$ := #{arg_name}_$
    let #{arg_name}_where_to_store_head$ := add(processed_return_value_init$, #{index * 32})
    return_value$ := #{yul_snippet_usage}
    #{Return.encode(type, "i$", "size$", "#{arg_name}_where_to_store_head$", "where_to_store_head_init$")}
    """
  end
\end{lstlisting}


\subsection{Return data encoding}
\label{ssec:return_data_encoding}

The Return data encoding module is responsible for encoding the value returned by an Elixireum function to an ABI format according to the typespec. It uses a similar approach to the calldata decoding module, but in the opposite way. The main function of this module is encode. Here is the base case of this function for simple types:

\begin{lstlisting}[language=elixir, caption={Encode base case}, label={lst:encode_base_case}]
  def encode(
    %Type{encoded_type: encoded_type, size: byte_size},
    _i_var_name,
    _size_var_name,
    where_to_store_head_var_name,
    _where_to_store_head_init_var_name
  )
  when encoded_type > 69 and encoded_type < 102 do
    offset = 8 * (32 - byte_size)

    """
    #{Utils.generate_type_check(...)}

    return_value$ := add(return_value$, 1)

    mstore(#{where_to_store_head_var_name}, shl(#{offset}, shr(#{offset}, mload(return_value$))))
    #{where_to_store_head_var_name} := add(#{where_to_store_head_var_name}, 32)

    return_value$ := add(return_value$, #{byte_size})
    """
  end
\end{lstlisting}

Here is an example of a recursive case:

\begin{lstlisting}[language=elixir, caption={Encode recursive case}, label={lst:encode_recursive_case}]
  def encode(
        %Type{
          encoded_type: 103 = encoded_type,
          items_count: size,
          components: [components]
        },
        i_var_name,
        size_var_name,
        where_to_store_head_var_name,
        where_to_store_head_init_var_name
      )
      when is_integer(size) do
    """
    #{Utils.generate_type_check(...)}

    return_value$ := add(return_value$, 1)
    let #{size_var_name} := mload(return_value$)

    #{Utils.generate_value_check(...)}

    return_value$ := add(return_value$, 32)

    for { let #{i_var_name} := 0 } lt(#{i_var_name}, #{size_var_name}) { #{i_var_name} := add(#{i_var_name}, 1) } {
      #{encode(components, i_var_name <> "_", size_var_name <> "_", where_to_store_head_var_name, where_to_store_head_init_var_name)}
    }
    """
  end
\end{lstlisting}
    


% \begin{longtable}{c|c}
% \caption[This is the title I want to appear in the List of Tables]{Simulation Parameters} \label{table:fousimulation_params} \\
% \hline
% A & B  \\
% \hline
% \endfirsthead
% \multicolumn{2}{@{}l}{} \\
% \hline
% A & B \\
% \hline
% \endhead
% \hline
%  \textbf{Parameter} & \textbf{Value}\\
%  \hline
%  Number of vehicles & $|\mathcal{V}|$\\
%  \hline
%  Number of RSUs & $|\mathcal{U}|$\\
%  \hline
%  RSU coverage radius & 150 m\\
%  \hline
%  V2V communication radius & 30 m\\
%  \hline
%  Smart vehicle antenna height & 1.5 m\\
%  \hline
%  RSU antenna height & 25 m\\
%  \hline
%  Smart vehicle maximum speed & $v_{max}$ m/s\\
%  \hline
%  Smart vehicle minimum speed & $v_{min}$ m/s\\
%  \hline
%  Common smart vehicle cache capacities & $[50, 100, 150, 200, 250]$ mb\\
%  \hline
%  Common RSU cache capacities & $[5000,1000,1500,2000,2500]$ mb\\
%  \hline
%  Common backhaul rates & $[75, 100, 150]$ mb/s\\
%  \hline
% \end{longtable}

% \begin{figure}[hbt]
% \centering
% \includegraphics[]{figs/inno.png}
% \caption{One kernel at $x_s$ (\emph{dotted kernel}) or two kernels at
% $x_i$ and $x_j$ (\textit{left and right}) lead to the same summed estimate
% at $x_s$. This shows a figure consisting of different types of
% lines. Elements of the figure described in the caption should be set in
% italics, in parentheses, as shown in this sample caption.}
% \label{fig:fouex}
% \end{figure}

% \ldots

\chapter{Results and Discussion}
\label{chap:res}


% 0. What/Why(purpose) Metrics 
%   - gas consumption
%     - on deploy
%     - on interacting
%   - compilation time
%   - ram/cpu usage at compilation time

% 1. test suite description (sources)

% \begin{lstlisting}[caption={ERC-20.exm}, language=elixir]
% defmodule ERC20 do
% ...
% end
% \end{lstlisting}

% \begin{lstlisting}[caption={ERC-721.exm}, language=elixir]
% defmodule ERC721 do
% ...
% end
% \end{lstlisting}

    
% 2. tests structure description
% - test suit works for all aforementioned contracts
%   - contract successfully compiles to EVM bytecode
%   - bytecode successfully deploys
%   - deployed contracts passes unit tests
% - gas consumption measurements
%     - on deploy
%     - on interacting
% - compilation time measurements
% - ram/cpu usage at compilation time measurements
% 3. test/metrics results one by one
% 4. results and comparison with Solidity
% 5. Discussion

% \subsection{Performance Metrics}
% Performance testing focuses on quantifying:
% \begin{enumerate}
%     \item \textbf{Gas Consumption}: Measured during both the deployment phase and subsequent contract interactions, providing insights into the cost efficiency of using Elixireum.
%     \item \textbf{Compilation Time}: The duration taken to compile contracts to bytecode, indicative of the compiler's efficiency.
%     \item \textbf{RAM/CPU Usage}: Monitored during the compilation process, reflecting the resource demands of the Elixireum compiler.
% \end{enumerate}

% This structured approach ensures a thorough evaluation of Elixireum, highlighting its capabilities and identifying areas for improvement. The subsequent section presents a detailed comparison of the test results, contrasting Elixireum's performance with Solidity to underscore its potential benefits and limitations.

This chapter presents an overview of results. Section \ref{sec:results_and_comparison} provides results we got during assessment of the Elixireum. Section \ref{sec:discussion} contains an interpretation of results.

\section{Results and Comparison}
\label{sec:results_and_comparison}
We conducted a series of tests by applying the test suite to a set of prepared smart contracts. All tests concluded successfully, signifying the achievement of the main project goal: Elixireum smart contracts successfully compile and deploy to the Ethereum blockchain. We then proceeded to compare Elixireum with Solidity. We benchmarked all the public functions of the ERC-20 implementation in Elixireum and its full analogue in Solidity, and conducted the same benchmarking for the ERC-721 implementation. Tables~\ref{tab:erc20_comparison} and \ref{tab:erc721_comparison} present the detailed metrics that we obtained.


\begin{table}[h!]
  \centering
  \renewcommand{\arraystretch}{1.2}
  \begin{tabular}{|c|c|c|}
  \hline
  \textbf{Method} & \textbf{Elixireum GasUsed} & \textbf{Solidity GasUsed} \\ \hline
  mint            & 58197.4              & 54799.6              \\ \hline
  approve         & 47619.8              & 46570.8              \\ \hline
  transferFrom    & 54812.0              & 48492.2              \\ \hline
  transfer        & 55659.8              & 51813.4              \\ \hline
  burn            & 33061.6              & 29433.25             \\ \hline
  deploy of contract       & 1939441              & 732543               \\ \hline
  \end{tabular}
  \caption{Gas usage comparison between Elixireum and Solidity for ERC-20 contract (20 runs)}
  \label{tab:erc20_comparison}
  \end{table}

\begin{table}[h!]
  \centering
  \renewcommand{\arraystretch}{1.2}
  \begin{tabular}{|c|c|c|}
  \hline
  \textbf{Method}         & \textbf{Elixireum GasUsed} & \textbf{Solidity GasUsed} \\ \hline
  setApprovalForAll       & 46596.0              & 46304.0              \\ \hline
  mint                    & 166817.4             & 162675.4             \\ \hline
  transferFrom            & 87123.0              & 55652.6              \\ \hline
  approve                 & 54798.0              & 49007.8              \\ \hline
  burn                    & 58539.6              & 29944.0              \\ \hline
  deploy of contract                 & 3067454              & 1113786              \\ \hline
  \end{tabular}
  \caption{Gas usage comparison between Elixireum and Solidity for ERC-721 contract (20 runs)}
  \label{tab:erc721_comparison}
  \end{table}

  The comparison reveals that while Elixireum successfully implements the functionalities of ERC-20 and ERC-721 contracts, it generally consumes more gas compared to Solidity for the same operations. Table~\ref{tab:erc20_comparison} shows that, on average, Elixireum is 8\% less efficient than Solidity for function calls of ERC-20 methods and approximately 164\% less efficient in terms of gas consumption during the deployment process. Table~\ref{tab:erc721_comparison} indicates even less favorable metrics: Elixireum is about 20.5\% less efficient for ERC-721 methods and consumes 175\% more gas during deployment. These findings indicate that while Elixireum is functional and compatible with the Ethereum blockchain, there is a need for optimization to reduce gas consumption and improve efficiency.

\section{Discussion}
\label{sec:discussion}

The findings from this chapter contribute significantly to the broader thesis goal of assessing the feasibility and potential of Elixireum in the context of smart contract development. We confirmed our hypothesis that an Elixir-like language can be compiled to EVM.

Moreover, we conducted a comprehensive comparison of gas consumption between Elixireum and Solidity. While the results showed that Elixireum consumes 2.7 times more gas than Solidity for deployment, the gas usage for ERC-20 and ERC-721 methods is comparable. This demonstrates that it is possible to have a functional language with dynamic typing and types stored in memory that consumes approximately 14\% more gas than Solidity.

With the rise of rollup chains mentioned in Chapter \ref{chap:intro}, the tradeoff between gas consumption and development ease should be considered. While a call of contract written with Elixireum may cost \$1.15 instead of \$1 with Solidity due to higher gas consumption, the ease and speed of development with Elixireum can lead to earlier deployment. This earlier deployment can attract more users, potentially outweighing the slight increase in gas costs.
\chapter{Conclusion}
\label{chap:concl}

We designed and implemented the Elixireum language along with its compiler. The Elixireum compiler and language are capable of supporting ERC-20 and ERC-721 standards, and we conducted measurements to compare our language with Solidity.

\section{Limitations}

The scope of this study was limited to the functionality that we considered most important to demonstrate the feasibility of Elixireum. Consequently, several features were left out of scope:

\begin{enumerate}
\item \textbf{External libraries:} External libraries are not implemented. This limitation prevents the reuse of the same functionality across multiple smart contracts and complicates the development of complex smart contracts, requiring redeployment of the same functionality and consequently requires more gas.
\item \textbf{External smart contract interaction:} Interaction with external smart contracts is not implemented. This limitation means that Elixireum cannot currently communicate with other contracts, making it impossible to implement popular patterns such as proxies or factories.
\item \textbf{Multiple files or modules:} Elixireum development is limited to a single file, making it impossible to reuse any Elixireum functionality from other smart contracts or to integrate additional functionalities from external sources.
\item \textbf{High deployment gas consumption:} The deployment of Elixireum contracts results in high gas consumption.
\item \textbf{Macros support:} We decided to exclude macros support from our initial scope due to the complexity of implementing this feature and time constraints.
\item \textbf{Pattern matching:} This feature was also excluded from the scope due to the similar reasons.
\end{enumerate}

\section{Future Work}

We propose the following directions for future work and development of Elixireum:

\begin{enumerate}
\item \textbf{Runtime mapping:} Implementing mapping runtime on top of the current storage module using EVM transient storage. Currently, Solidity does not support mappings in memory, only in storage, so this feature could be highly innovative and novel.
\item \textbf{Reducing deployment gas consumption:} The high gas consumption of deployment is likely connected to the large amount of generated Yul code. We believe it is possible to reduce these costs by factoring out similar Yul code snippets into standard functions.
\end{enumerate}




%% REFERENCES
\printbibliography[heading=bibintoc,title={Bibliography cited}]
% \appendix
\titleformat{\section}[hang]{\fontsize{20}{24}\selectfont\filcenter}{\thechapter.\arabic{section}}{1em}{}
\chapter{Source code}
\section{copy\_from\_pointer\_to\$}
\label{appex:copy_from_pointer_to}
  \begin{lstlisting}[language=elixir]
    function copy_from_pointer_to$(ptr_from, ptr_to) -> ptr_from_end, ptr_to_end {
      let type := byte(0, mload(ptr_from))
      mstore8(ptr_to, type)
      ptr_to := add(ptr_to, 1)
      ptr_from := add(ptr_from, 1)
      let size := 0

      switch type
      case 1 {
        size := mload(ptr_from)
        mstore(ptr_to, size)
        ptr_to := add(ptr_to, 32)
        ptr_from := add(ptr_from, 32)

        let iterations_count := add(1, sdiv(sub(size, 1), 32))

        for {let i := 0} lt(i, iterations_count) {i := add(i, 1)} {
          mstore(ptr_to, mload(ptr_from))
          ptr_from := add(ptr_from, 32)
          ptr_to := add(ptr_to, 32)
        }
        let mod_ := mod(size, 32)
        if gt(mod_, 0) {
          let dif := sub(32, mod_)
          ptr_to_end := sub(ptr_to, dif)
          ptr_from_end := sub(ptr_from, dif)
        }
        leave
      }

      case 102 {
        size := mload(ptr_from)
        mstore(ptr_to, size)
        ptr_to := add(ptr_to, 32)
        ptr_from := add(ptr_from, 32)

        let iterations_count := add(1, sdiv(sub(size, 1), 32))

        for {let i := 0} lt(i, iterations_count) {i := add(i, 1)} {
          mstore(ptr_to, mload(ptr_from))
          ptr_from := add(ptr_from, 32)
          ptr_to := add(ptr_to, 32)
        }
        let mod_ := mod(size, 32)
        ptr_to_end := ptr_to
        ptr_from_end := ptr_from
        if gt(mod_, 0) {
          let dif := sub(32, mod_)
          ptr_to_end := sub(ptr_to, dif)
          ptr_from_end := sub(ptr_from, dif)
        }
        leave
      }

      case 2 {
        size := 1
      }

      case 68 {
        size := 20
      }

      case 3 {
        // tuple
        size := mload(ptr_from)
        ptr_from := add(ptr_from, 32)
        mstore(ptr_to, size)
        ptr_to := add(ptr_to, 32)

        for {let i := 0} lt(i, size) {i := add(i, 1)} {
          ptr_from, ptr_to := copy_from_pointer_to$(ptr_from, ptr_to)
        }
        leave
      }

      case 103 {
        // list
        size := mload(ptr_from)
        ptr_from := add(ptr_from, 32)
        mstore(ptr_to, size)
        ptr_to := add(ptr_to, 32)

        for {let i := 0} lt(i, size) {i := add(i, 1)} {
          ptr_from, ptr_to := copy_from_pointer_to$(ptr_from, ptr_to)
        }
        leave
      }


      default {
        if lt(type, 102) {
          size := sub(type, 69)
        }
        if lt(type, 68) {
          size := sub(type, 35)
        }
        if lt(type, 36) {
          size := sub(type, 3)
        }
      }

      mstore(ptr_to, mload(ptr_from))
      ptr_from_end := add(ptr_from, size)
      ptr_to_end := add(ptr_to, size)
    }
  \end{lstlisting}
    
\end{document}


\chapter{Literature Review}
\label{chap:lr}
\chaptermark{Second Chapter Heading}


\section{Ethereum Virtual Machine}
Ethereum is a decentralized system, and at its core, the Ethereum Virtual Machine (EVM) is an essential component of the Ethereum blockchain. The EVM serves as the computational heart of the Ethereum network, enabling the execution of smart contracts and providing the foundation for a wide array of decentralized applications.

In the context of smart contract execution, it's important to mention the concept of gas limit. Gas limit is a critical aspect of Ethereum's execution model. Each operation and computation on the Ethereum Virtual Machine consumes a certain amount of gas, which is a measure of computational work. The gas limit is a cap set by the user or the entity initiating a contract execution, and it represents the maximum amount of gas they are willing to spend for that operation.

Gas limit, the EVM's stack-based architecture, and the wide set of available operations allow it to efficiently and securely for miners execute these smart contracts, granting it the status of a Turing-complete machine \cite{EthereumWhitepaper}. This level of computational flexibility empowers developers to create sophisticated programs.

\section{Existing EVM languages}
According to the official Ethereum documentation \cite{OfficialEthereumLanguages} only four languages for EVM exist: Solidity, Vyper, Yul and Fe. However, other attempts to create language for EVM are known \cite{CommunityEthereumLanguages}. Although there are many EVM languages, according to sOmEtHiNg only two of them are widely used: Solidity and Vyper.

\subsection{Solidity}

\subsection{Vyper}

\section{Market Research}

\section{Elixir}

\section{Compiler construction}

\section{Conclusion}

% \newpage


% \begin{longtable}{c|c}
% \caption[This is the title I want to appear in the List of Tables]{Simulation Parameters} \label{table:secsimulation_params} \\
% \hline
% A & B  \\
% \hline
% \endfirsthead
% \multicolumn{2}{@{}l}{} \\
% \hline
% A & B \\
% \hline
% \endhead
% \hline
%  \textbf{Parameter} & \textbf{Value}\\
%  \hline
%  Number of vehicles & $|\mathcal{V}|$\\
%  \hline
%  Number of RSUs & $|\mathcal{U}|$\\
%  \hline
%  RSU coverage radius & 150 m\\
%  \hline
%  V2V communication radius & 30 m\\
%  \hline
%  Smart vehicle antenna height & 1.5 m\\
%  \hline
%  RSU antenna height & 25 m\\
%  \hline
%  Smart vehicle maximum speed & $v_{max}$ m/s\\
%  \hline
%  Smart vehicle minimum speed & $v_{min}$ m/s\\
%  \hline
%  Common smart vehicle cache capacities & $[50, 100, 150, 200, 250]$ mb\\
%  \hline
%  Common RSU cache capacities & $[5000,1000,1500,2000,2500]$ mb\\
%  \hline
%  Common backhaul rates & $[75, 100, 150]$ mb/s\\
%  \hline
% \end{longtable}

% \begin{figure}[hbt]
% \centering
% \includegraphics[]{figs/inno.png}
% \caption{One kernel at $x_s$ (\emph{dotted kernel}) or two kernels at
% $x_i$ and $x_j$ (\textit{left and right}) lead to the same summed estimate
% at $x_s$. This shows a figure consisting of different types of
% lines. Elements of the figure described in the caption should be set in
% italics, in parentheses, as shown in this sample caption.}
% \label{fig:secex}
% \end{figure}

% This description implies several essential properties of the task at hand:
% \begin{enumerate}
%     \item Watermark must contain all necessary information, but still, be placeable and recognizable even on smaller images. The produced watermark must be compact but have the possibility to store enough information. 
%     \item To prevent easy tampering, the watermark must be invisible to the naked eye (and, preferably, to basic image parsing tools). If malefactor does not know about the existence of watermark, they might not even try to remove it and disable it. 
% \end{enumerate}

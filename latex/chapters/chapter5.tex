\chapter{Results and Discussion}
\label{chap:res}


% 0. What/Why(purpose) Metrics 
%   - gas consumption
%     - on deploy
%     - on interacting
%   - compilation time
%   - ram/cpu usage at compilation time

% 1. test suite description (sources)

% \begin{lstlisting}[caption={ERC-20.exm}, language=elixir]
% defmodule ERC20 do
% ...
% end
% \end{lstlisting}

% \begin{lstlisting}[caption={ERC-721.exm}, language=elixir]
% defmodule ERC721 do
% ...
% end
% \end{lstlisting}

    
% 2. tests structure description
% - test suit works for all aforementioned contracts
%   - contract successfully compiles to EVM bytecode
%   - bytecode successfully deploys
%   - deployed contracts passes unit tests
% - gas consumption measurements
%     - on deploy
%     - on interacting
% - compilation time measurements
% - ram/cpu usage at compilation time measurements
% 3. test/metrics results one by one
% 4. results and comparison with Solidity
% 5. Discussion

% \subsection{Performance Metrics}
% Performance testing focuses on quantifying:
% \begin{enumerate}
%     \item \textbf{Gas Consumption}: Measured during both the deployment phase and subsequent contract interactions, providing insights into the cost efficiency of using Elixireum.
%     \item \textbf{Compilation Time}: The duration taken to compile contracts to bytecode, indicative of the compiler's efficiency.
%     \item \textbf{RAM/CPU Usage}: Monitored during the compilation process, reflecting the resource demands of the Elixireum compiler.
% \end{enumerate}

% This structured approach ensures a thorough evaluation of Elixireum, highlighting its capabilities and identifying areas for improvement. The subsequent section presents a detailed comparison of the test results, contrasting Elixireum's performance with Solidity to underscore its potential benefits and limitations.

This chapter presents an overview of results. Section \ref{sec:results_and_comparison} provides results we got during assessment of the Elixireum. Section \ref{sec:discussion} contains an interpretation of results.

\section{Results and Comparison}
\label{sec:results_and_comparison}
We conducted a series of tests by applying the test suite to a set of prepared smart contracts. All tests concluded successfully, signifying the achievement of the main project goal: Elixireum smart contracts successfully compile and deploy to the Ethereum blockchain. We then proceeded to compare Elixireum with Solidity. We benchmarked all the public functions of the ERC-20 implementation in Elixireum and its full analogue in Solidity, and conducted the same benchmarking for the ERC-721 implementation. Tables~\ref{tab:erc20_comparison} and \ref{tab:erc721_comparison} present the detailed metrics that we obtained.


\begin{table}[h!]
  \centering
  \renewcommand{\arraystretch}{1.2}
  \begin{tabular}{|c|c|c|}
  \hline
  \textbf{Method} & \textbf{Elixireum Gas Used} & \textbf{Solidity Gas Used} \\ \hline
  mint            & 58197.4              & 54799.6              \\ \hline
  approve         & 47619.8              & 46570.8              \\ \hline
  transferFrom    & 54812.0              & 48492.2              \\ \hline
  transfer        & 55659.8              & 51813.4              \\ \hline
  burn            & 33061.6              & 29433.25             \\ \hline
  deploy of contract       & 1939441              & 732543               \\ \hline
  \end{tabular}
  \caption{Gas usage comparison between Elixireum and Solidity for ERC-20 contract (20 runs)}
  \label{tab:erc20_comparison}
  \end{table}

\begin{table}[h!]
  \centering
  \renewcommand{\arraystretch}{1.2}
  \begin{tabular}{|c|c|c|}
  \hline
  \textbf{Method}         & \textbf{Elixireum Gas Used} & \textbf{Solidity Gas Used} \\ \hline
  setApprovalForAll       & 46596.0              & 46304.0              \\ \hline
  mint                    & 166817.4             & 162675.4             \\ \hline
  transferFrom            & 87123.0              & 55652.6              \\ \hline
  approve                 & 54798.0              & 49007.8              \\ \hline
  burn                    & 58539.6              & 29944.0              \\ \hline
  deploy of contract                 & 3067454              & 1113786              \\ \hline
  \end{tabular}
  \caption{Gas usage comparison between Elixireum and Solidity for ERC-721 contract (20 runs)}
  \label{tab:erc721_comparison}
  \end{table}

  The comparison reveals that while Elixireum successfully implements the functionalities of ERC-20 and ERC-721 contracts, it generally consumes more gas compared to Solidity for the same operations. Table~\ref{tab:erc20_comparison} shows that, on average, Elixireum is 8\% less efficient than Solidity for function calls of ERC-20 methods and approximately 164\% less efficient in terms of gas consumption during the deployment process. Table~\ref{tab:erc721_comparison} indicates even less favorable metrics: Elixireum is about 20.5\% less efficient for ERC-721 methods and consumes 175\% more gas during deployment. These findings indicate that while Elixireum is functional and compatible with the Ethereum blockchain, there is a need for optimization to reduce gas consumption and improve efficiency.

\section{Discussion}
\label{sec:discussion}

The findings from this chapter contribute significantly to the broader thesis goal of assessing the feasibility and potential of Elixireum in the context of smart contract development. We confirmed our hypothesis that an Elixir-like language can be compiled to EVM.

Moreover, we conducted a comprehensive comparison of gas consumption between Elixireum and Solidity. While the results showed that Elixireum consumes 2.7 times more gas than Solidity for deployment, the gas usage for ERC-20 and ERC-721 methods is comparable. This demonstrates that it is possible to have a functional language with dynamic typing and types stored in memory that consumes approximately 14\% more gas than Solidity.

With the rise of rollup chains mentioned in Chapter \ref{chap:intro}, the tradeoff between gas consumption and development ease should be considered. While a call of contract written with Elixireum may cost \$1.15 instead of \$1 with Solidity due to higher gas consumption, the ease and speed of development with Elixireum can lead to earlier deployment. This earlier deployment can attract more users, potentially outweighing the slight increase in gas costs.
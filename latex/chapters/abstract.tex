\begin{abstract}

This thesis addresses the necessity for a new smart contract programming language within the Ethereum ecosystem, currently dominated by contracts written in Solidity. This dominance results in monolingualism\footnote{Monolingualism - This term refers to the use of a single language by an individual or community, without the regular use of any additional languages.} and monoparadigmism\footnote{Monoparadigmism -  This term describes the use of only one programming paradigm within a programming environment or by a programmer, excluding the use of multiple paradigms.}. Our research introduces Elixireum, a functional language featuring Elixir syntax and dynamic typing, designed to enhance the flexibility and creativity of decentralized application (dApp) developers. The Elixireum compiler was developed and tested for compatibility with the Ethereum Virtual Machine (EVM) and its ability to support ERC-20 and ERC-721 standards. A comprehensive test suite was employed to compare the gas consumption of Elixireum and Solidity. The results revealed that, although Elixireum is functional and compatible with Ethereum, it consumes more gas than Solidity. Key features, such as events and storage interaction, were implemented. However, limitations include high deployment gas consumption, the absence of external libraries, and interaction with external smart contracts. Future work should focus on reducing deployment gas costs and implementing runtime mapping. Despite the higher gas consumption, the ease of development on Elixireum and the potential for earlier deployment could outweigh these costs, positioning it as a viable alternative in the Web3 space.
\end{abstract}
\chapter{Выводы}
\label{chap:conclusion}

\section{Результаты и сравнение}
\label{sec:results_and_comparison}
Мы провели серию тестов, применив тестовый пакет к набору подготовленных смарт-контрактов. Все тесты завершились успешно, что свидетельствует о достижении главной цели проекта: разработан компилятор, способный транслировать язык Elixireum в байткод EVM. Кроме того, мы развернули токен ERC-20, реализованный на языке Elixireum, в тестовой сети Sepolia, следующие шестнадцатеричные байты - адрес развернутого токена 0x9DC699F6F8F3E42D4C6ae368C751325dC4106279\footnote{\url{https://eth-sepolia.blockscout.com/token/0x9dc699f6f8f3e42d4c6ae368c751325dc4106279}}. Затем мы приступили к сравнению Elixireum с Solidity. Мы сравнили публичные функции реализации ERC-20 в Elixireum и ее полного аналога в Solidity, а также провели аналогичный бенчмарк для реализации ERC-721. В таблицах \ref{tab:erc20_comparison} и \ref{tab:erc721_comparison} представлены подробные метрики, которые мы получили.

\begin{table}[h!]
  \centering
  \renewcommand{\arraystretch}{1.2}
  \begin{tabular}{|c|c|c|}
  \hline
  \textbf{Method} & \textbf{Elixireum Gas Used} & \textbf{Solidity Gas Used} \\ \hline
  mint            & 58197.4              & 54799.6              \\ \hline
  approve         & 47619.8              & 46570.8              \\ \hline
  transferFrom    & 54812.0              & 48492.2              \\ \hline
  transfer        & 55659.8              & 51813.4              \\ \hline
  burn            & 33061.6              & 29433.25             \\ \hline
  deploy of contract       & 1939441              & 732543               \\ \hline
  \end{tabular}
  \caption{Сравнение расхода газа по контракту ERC-20 (20 запусков)}
  \label{tab:erc20_comparison}
  \end{table}
  
  \begin{table}[h!]
  \centering
  \renewcommand{\arraystretch}{1.2}
  \begin{tabular}{|c|c|c|}
  \hline
  \textbf{Method}         & \textbf{Elixireum Gas Used} & \textbf{Solidity Gas Used} \\ \hline
  setApprovalForAll       & 46596.0              & 46304.0              \\ \hline
  mint                    & 166817.4             & 162675.4             \\ \hline
  transferFrom            & 87123.0              & 55652.6              \\ \hline
  approve                 & 54798.0              & 49007.8              \\ \hline
  burn                    & 58539.6              & 29944.0              \\ \hline
  deploy of contract                 & 3067454              & 1113786              \\ \hline
  \end{tabular}
  \caption{Сравнение расхода газа по контракту ERC-721 (20 запусков)}
  \label{tab:erc721_comparison}
  \end{table}

  Сравнение показывает, что хотя Elixireum успешно реализует функциональность контрактов ERC-20 и ERC-721, он в целом потребляет больше газа по сравнению с Solidity для выполнения тех же операций. Таблица~\ref{tab:erc20_comparison} показывает, что в среднем Elixireum на 8\% менее эффективен, чем Solidity, для вызовов функций методов ERC-20 и примерно на 164\% менее эффективен с точки зрения потребления газа в процессе развертывания. Таблица~\ref{tab:erc721_comparison} показывает еще менее благоприятные показатели: Elixireum примерно на 20,5\% менее эффективен для методов ERC-721 и потребляет на 175\% больше газа во время развертывания. Эти данные свидетельствуют о том, что, несмотря на функциональность и совместимость Elixireum с блокчейном Ethereum, существует необходимость в оптимизации для снижения потребления газа и повышения эффективности.
  
\section{Дискуссия}
\label{sec:discussion}

Результаты, полученные в этой главе, вносят значительный вклад в достижение более широкой цели работы - оценки осуществимости и потенциала Elixireum в контексте разработки смарт-контрактов. Мы подтвердили нашу гипотезу о том, что Elixir-подобный язык может быть скомпилирован в EVM.

Кроме того, мы провели комплексное сравнение потребления газа между Elixireum и Solidity. Результаты показали, что Elixireum потребляет в 2,7 раза больше газа, чем Solidity при развертывании, однако расход газа для методов ERC-20 и ERC-721 сопоставим. Это показывает, что можно иметь функциональный язык с динамической типизацией и типами, хранящимися в памяти, который потребляет примерно на 14\% больше газа, чем Solidity.

С ростом цепочек свертывания, следует рассмотреть компромисс между потреблением газа и простотой разработки. Хотя вызов контракта, написанный в Elixireum, может стоить \$1,15 вместо \$1 в Solidity из-за большего потребления газа, простота и скорость разработки в Elixireum может привести к более раннему развертыванию. Такое более раннее внедрение может привлечь больше пользователей, что потенциально перевесит небольшое увеличение расходов на газ.
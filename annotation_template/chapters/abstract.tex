\begin{abstract}
% skip one line to make the abstract start with indent

В данной диссертации рассматривается необходимость создания нового языка программирования смарт-контрактов в экосистеме Ethereum, в которой в настоящее время доминируют контракты, написанные на Solidity. Это доминирование приводит к таким ограничениям, как монолингвизм\footnote{Монолингвизм - Этот термин относится к использованию одного языка отдельным человеком или сообществом, без регулярного использования каких-либо дополнительных языков.} и монопарадигмизм\footnote{Монопарадигмизм - Этот термин описывает использование только одной парадигмы программирования в среде программирования или программистом, исключая использование нескольких парадигм.}. В нашем исследовании представлен Elixireum - функциональный язык с синтаксисом Elixir и динамической типизацией, предназначенный для повышения гибкости и креативности разработчиков децентрализованных приложений (dApp). Компилятор Elixireum был разработан и протестирован на совместимость с виртуальной машиной Ethereum (EVM) и способность поддерживать стандарты ERC-20 и ERC-721. Для сравнения потребления газа Elixireum и Solidity был использован комплексный набор тестов. Результаты показали, что, несмотря на функциональность и совместимость Elixireum с Ethereum, он потребляет больше газа, чем Solidity. Ключевые функции, такие как события и взаимодействие с хранилищем, были реализованы. Однако среди ограничений - высокий расход газа на развертывание, отсутствие внешних библиотек и взаимодействия с внешними смарт-контрактами. Будущая работа должна быть направлена на снижение затрат газа при развертывании и реализацию словарей, которые бы работали во время выполнения. Несмотря на высокое потребление газа, простота разработки и потенциал для более раннего развертывания Elixireum могут перевесить эти затраты, позиционируя его как жизнеспособную альтернативу в пространстве Web3.
\end{abstract}
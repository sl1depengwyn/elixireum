\chapter{Обзор литературы}
\label{chap:lr}
\chaptermark{Второй заголовок главы}

\section{Существующие языки EVM}
\label{sec:langs}
Официальная документация Ethereum \cite{OfficialEthereumLanguages} перечисляет только четыре языка, которые компилируются в EVM: Solidity, Vyper, Yul и Fe. Кроме того, другие попытки создать язык для EVM известны\footnote{\href{https://github.com/s-tikhomirov/smart-contract-languages}{https://github.com/s-tikhomirov/smart-contract-languages} -- Коллекция ресурсов по языкам программирования смарт-контрактов}, но в данной статье будут рассмотрены только официально задокументированные языки.



Среди доступных языков для EVM, Solidity выделяется как наиболее широко принятый и используемый \cite{SolidityWidelyUsed, SolidityDevsChallenges}. Это доминирование еще больше подчеркивается данными, представленными в статье\footnote{\href{https://blockchain.oodles.io/blog/solidity-vyper-smart-contracts-programming/}{https://blockchain.oodles.io/blog/solidity-vyper-smart-contracts-programming/} -- Анализ Solidity и Vyper для программирования смарт-контрактов}, из которой следует, что на Github файлов Solidity примерно в 800 раз больше, чем файлов Vyper. Следовательно, большинство существующих смарт-контрактов на блокчейне Ethereum написаны на Solidity. С одной стороны, концентрация разработок и ресурсов вокруг одного языка, такого как Solidity, может рассматриваться положительно. Это способствует формированию сильного чувства сплоченности сообщества и гарантирует, что большинство разработчиков сосредоточены на доработке и улучшении одного языка, что потенциально повышает его надежность и расширяет набор функций. Однако такое доминирование порождает и серьезные проблемы. Широкая распространенность Solidity в экосистеме EVM означает, что если в языке Solidity возникает уязвимость или недостаток безопасности, то это затрагивает значительную часть смарт-контрактов в сети. Такая централизация может быть обоюдоострым мечом, поскольку одна точка сбоя в Solidity может иметь далеко идущие последствия для всего блокчейна Ethereum. Кроме того, сообщество Ethereum критикует Solidity. Члены сообщества высказывают свои опасения и проблемы, связанные с языком \cite{SolidityDevsChallenges}. Эти проблемы варьируются от сложности языка до отсутствия в нем определенных функций, что может препятствовать разработке надежных и безопасных смарт-контрактов. Таким образом, децентрализация и возможность выбора очень важны, особенно для такой децентрализованной экосистемы, как блокчейн. Вот краткое введение в самые популярные языки смарт-контрактов: Solidity и Vyper.

\subsection{Solidity}

Solidity - это высокоуровневый, объектно-ориентированный язык программирования, статически типизированный, специально разработанный для EVM. Его синтаксис похож на C++, Python и Javascript \cite{SolidityInspirasion}. Он поддерживает множественное наследование, сложные пользовательские типы, библиотеки и некоторые другие возможности \cite{SolidityFeatures}.

\subsection{Vyper}

Vyper значительно отличается от Solidity. Он основан на Python и расширяет его для разработки смарт-контрактов. Vyper стремится к безопасности и простоте в плане реализации самого компилятора и читаемости кода для более легкого аудита. Например, в нем отсутствует список функций, которые считаются вредными для читаемости кода, что делает язык более склонным к ошибкам \cite{VyperDescription}.

\section{Язык Elixir}
\label{sec:ex}

Elixir \cite{ElixirOfficialWebSite} - это современный язык программирования, работающий на виртуальной машине Erlang. Благодаря поддержке неизменяемых переменных Elixir значительно повышает ясность и простоту кода смарт-контрактов. Эта неизменяемость гарантирует, что после определения данных они не могут быть изменены, что способствует предсказуемости и безопасности.

Отличительной особенностью Elixir являются возможности метапрограммирования. Эта функциональность позволяет разработчикам создавать языки, специфичные для конкретной области. Эта возможность бесценна, так как позволяет разработчикам выражать логику программы в удобном для чтения виде и в тесной связи с конкретной проблемной областью. Благодаря использованию метапрограммирования разработка смарт-контрактов станет более простой и доступной для широкого круга разработчиков.

Как утверждает О'Грейди \cite{RedMonk}, Elixir уверенно держится в списке 50 лучших языков программирования. Примечательно, что он превосходит по популярности даже Solidity, что делает его привлекательным выбором для разработчиков Ethereum и тем самым способствует расширению сообщества Ethereum.

Подводя итог, можно сказать, что наше решение принять Elixir в качестве языка для нашего компилятора и основополагающего фреймворка для языка Elixireum подкреплено его неизменяемостью, возможностями метапрограммирования и известностью в сфере разработки программного обеспечения. Этот выбор обеспечивает простоту и безопасность разработки смарт-контрактов, а также рост и диверсификацию экосистемы Ethereum.


\section{Заключение}
\label{sec:conc}

В заключение этой главы следует отметить два важных момента. Во-первых, доминирование Solidity создало проблемы для экосистемы Ethereum. Во-вторых, отсутствие функциональных языков для разработки смарт-контрактов ограничивает использование преимуществ функциональной парадигмы. Поэтому мы решили разработать Elixireum, функциональный язык программирования, построенный на основе Elixir и специально предназначенный для разработки смарт-контрактов.